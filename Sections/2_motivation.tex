\section{Motivation for Spoofing User Data}
\label{sec:motivation}

The motivation for our work stems from the realization that current approaches for protecting user data in the Android ecosystem are insufficient leaving user data exposed to privacy violations from the installed apps. 

\subsection{Inadequate Android Permission Framework}

The Android permission framework allows users to control app data access to protect sensitive info but doesn't prevent unauthorized data collection. Researchers~\cite{hasan2013sensing, simon2013pin, ba2020learning, shen2015input, lin2019motion} highlight that ``normal permissions'' (e.g., inertial sensors) can reveal sensitive data, yet Android grants access without notifying users.

``Dangerous permissions'', like location, require explicit user consent. Granting access may cause privacy risks, while denial triggers a \texttt{SecurityException}, leading to crashes or restricting app functionality. Wagner et al.~\cite{wijesekera2017feasibility} found only 17\% of users consider permissions when installing apps. Recent options, such as ``\textit{Only this time}'' and ``\textit{While using the app},'' enhance privacy by limiting background access, but the framework remains inflexible, offering users few choices.


\subsection{Limitations of Existing User Data Spoofing Mechanisms}

The limitations of Android's permission framework have led researchers to explore deceiving app-accessed user data. Modifying the Android OS or target apps~\cite{backes2015boxify, jeon2012dr, raval2016you, smalley2013security, wu2017context} requires a custom ROM, which demands root access, making it impractical. Additionally, altering each app's source code is not scalable and can be detected using Google's Play Integrity API~\cite{andPlayIntAPI}.


\subsection{Spoofing: Achilles Heel of Continuous User Authentication}
\label{sec:continuous_authentication_mechanisms}

% \begin{table}[h]
    {\centering
    \begin{tabular}{>{\centering\arraybackslash}m{2.5cm} > {\centering\arraybackslash}m{3cm} >{\centering\arraybackslash}m{4cm} >{\centering\arraybackslash}m{2.5cm}}
        \hline
          \textbf{Modality} & \textbf{Framework} & \textbf{User Data} & \textbf{Accuracy} \\
         \hline
         \textbf{Gait} \\
         & ~\cite{kolokas2019gait, sun2018artificial, thang2012gait, hoang2013adaptive} & Ac, Ca, Gy, Mg & 91-95\% \\
         \hline
         
         \textbf{Gesture} \\
         Flick & ~\cite{shih2015flick, nohara2016personal} & Ac, Gy & 92.8-98\% \\
         
         Swipe & ~\cite{lu2015safeguard, jain2015exploring} & Ac, Or, To \\
         
         Touch & ~\cite{nixon2016slowmo, feng2014tips} & To & 89\%\\
         \hline

         \textbf{Motion} \\
         Free & ~\cite{abuhamad2020autosen, amini2018deepauth, li2018using} & Ac, Gy, Mg & 96.7-97.5\% \\
         
         Shake & ~\cite{yan2018towards} & Ac, Gy & 96.87\%\\

         Eye & ~\cite{song2016eyeveri} & Ca & 88.73\% \\

         IaH & ~\cite{xia2018motionhacker} & Ac, Gy & 32.8\% \\
         
         Gesture & ~\cite{hong2016mgra, hong2015waving} & Ac & 92.2-95.8\%\\

         \hline

         
         
         \textbf{Voice} \\
         & ~\cite{miguel2016interaction, zhang2016voicelive, wang2019voicepop, johnson2013secure} & Mi & 93.5-99.3\% \\
         \hline 
        
         \textbf{Multimodal} \\
         Ge, Mo & ~\cite{khamis2016gazetouchpass, zhu2013sensec} & Ac, Gy, Or, To & 65\% \\
         
         Ga, Ge, Mo & ~\cite{sitova2015hmog} & Ac, Gy, To & \\

         Be, Ga, Ge & ~\cite{pang2019mineauth, acien2019multilock} & Ac, Nw, Tk, To & 85-97.1\% \\
         
         Ga, Ge & ~\cite{zhu2019riskcog, lee2017implicit} & Ac, Gy, Li, Mg & 95.6-98.1\% \\

        \hline
        \multicolumn{4}{p{12cm}}{\footnotesize \textbf{Ac}: Accelerometer, \textbf{Ca}: Camera,  \textbf{Gr}: Gravity Sensor, \textbf{Gy}: Gyroscope, \textbf{Li}: Light Sensor, \textbf{Mg}: Magnetometer,  \textbf{Mi}: Microphone, \textbf{Nw}: Network, \textbf{Or}: Orientation, \textbf{To}: Touch, \textbf{Tr}: Tracking} \\
        \hline
        
        \multicolumn{4}{p{12cm}}{\footnotesize \textbf{Be}: Behaviour, \textbf{Ga}: Gait, \textbf{Ge}: Gesture, \textbf{IaH}: In-air Handwriting, \textbf{Mo}: Motion}\\
        \hline
    \end{tabular}
    }
    \caption{Existing continuous authentication methods.}
    \label{tab:ca_approaches_deceivable}
\end{table}
Knowledge-based authentication, requiring passwords for access, is simple but suffers from frequent re-entry and reuse risks. To address these issues, continuous authentication verifies identity using behavioral biometrics (e.g., keystroke dynamics, touch gestures, motion, voice), leveraging inherent user signatures to prevent unauthorized access.

Biometric signatures are captured through interaction, environmental, and sensor data, assuming uniqueness and resistance to spoofing. However, most authentication mechanisms rely on sensor data, making them vulnerable to attacks where false sensor inputs mimic legitimate user behavior, enabling unauthorized access.

Despite this vulnerability, no robust user data spoofing mechanism exists, making such attacks difficult in practice. Consequently, weak continuous authentication methods remain widely accepted~\cite{sun2018artificial, shih2015flick, jain2015exploring, feng2014tips, li2018using, yan2018towards, song2016eyeveri, xia2018motionhacker, hong2016mgra, johnson2013secure, zhu2013sensec, sitova2015hmog, pang2019mineauth, lee2017implicit}.

To overcome the limitations of the Android permission framework and provide a benchmark for continuous authentication, a robust data spoofing mechanism is needed. This paper addresses this gap by analyzing and improving existing spoofing techniques.
