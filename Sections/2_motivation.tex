\section{Motivation for Spoofing User Data}
\label{sec:motivation}

The motivation for our work stems from the realization that current approaches for protecting user data in the Android ecosystem are insufficient leaving user data exposed to privacy violations from the installed apps. 

\subsection{Inadequate Android Permission Framework}

The Android platform provides a permission framework that enables the user to control the data access privileges granted to individual apps. It aims to safeguard sensitive user data by restricting unauthorized access. However, it has been found to be inadequate failing to prevent apps from collecting sensitive data. Researchers~\cite{hasan2013sensing, simon2013pin, ba2020learning, shen2015input, lin2019motion} have shown that non-runtime permissions, also referred to as ``normal permissions'', like those related to the inertial measurement unit, can contain sensitive information, but the Android permission framework grants access to them without notifying the user.

For runtime permissions, also referred to as ``dangerous permissions'', like location, requested by apps, the user is presented with a binary choice to either grant or deny each permission. Granting a requested permission can result in a privacy breach while denying it triggers a \texttt{SecurityException} in the app. If the exception is not handled properly, the app crashes. Even if the exception is handled correctly, due to a lack of permission, the app limits its services. Further, Wagner et al.\cite{wijesekera2017feasibility} noted that only 17\% of users pay attention to permissions during installation. Recently added options, such as ``\textit{Only this time}'' and ``\textit{While using the app}'', in the permission framework aim to protect user privacy by preventing background permission usage. Hence, the existing permission framework lacks flexibility, leaving users with limited choices.

\subsection{Limitations of Existing User Data Spoofing Mechanisms}

The aforementioned inadequacy of the Android permission framework has motivated researchers to look for an alternative option, i.e., to deceive the user data fed to the apps. While modifying the source code of the Android OS or target apps~\cite{backes2015boxify, jeon2012dr, raval2016you, smalley2013security, wu2017context} might seem like a solution, this approach necessitates the creation of a custom ROM. However, installing custom ROMs on Android devices requires root privileges, making it impractical for most users. Additionally, modifying the source code of each target app is not a scalable solution. This mechanism can also be easily detected if Google's Play Integrity API is utilized in the target app~\cite{andPlayIntAPI}.

% \begin{table}[h]
    {\centering
    \begin{tabular}{>{\centering\arraybackslash}m{2.5cm} > {\centering\arraybackslash}m{3cm} >{\centering\arraybackslash}m{4cm} >{\centering\arraybackslash}m{2.5cm}}
        \hline
          \textbf{Modality} & \textbf{Framework} & \textbf{User Data} & \textbf{Accuracy} \\
         \hline
         \textbf{Gait} \\
         & ~\cite{kolokas2019gait, sun2018artificial, thang2012gait, hoang2013adaptive} & Ac, Ca, Gy, Mg & 91-95\% \\
         \hline
         
         \textbf{Gesture} \\
         Flick & ~\cite{shih2015flick, nohara2016personal} & Ac, Gy & 92.8-98\% \\
         
         Swipe & ~\cite{lu2015safeguard, jain2015exploring} & Ac, Or, To \\
         
         Touch & ~\cite{nixon2016slowmo, feng2014tips} & To & 89\%\\
         \hline

         \textbf{Motion} \\
         Free & ~\cite{abuhamad2020autosen, amini2018deepauth, li2018using} & Ac, Gy, Mg & 96.7-97.5\% \\
         
         Shake & ~\cite{yan2018towards} & Ac, Gy & 96.87\%\\

         Eye & ~\cite{song2016eyeveri} & Ca & 88.73\% \\

         IaH & ~\cite{xia2018motionhacker} & Ac, Gy & 32.8\% \\
         
         Gesture & ~\cite{hong2016mgra, hong2015waving} & Ac & 92.2-95.8\%\\

         \hline

         
         
         \textbf{Voice} \\
         & ~\cite{miguel2016interaction, zhang2016voicelive, wang2019voicepop, johnson2013secure} & Mi & 93.5-99.3\% \\
         \hline 
        
         \textbf{Multimodal} \\
         Ge, Mo & ~\cite{khamis2016gazetouchpass, zhu2013sensec} & Ac, Gy, Or, To & 65\% \\
         
         Ga, Ge, Mo & ~\cite{sitova2015hmog} & Ac, Gy, To & \\

         Be, Ga, Ge & ~\cite{pang2019mineauth, acien2019multilock} & Ac, Nw, Tk, To & 85-97.1\% \\
         
         Ga, Ge & ~\cite{zhu2019riskcog, lee2017implicit} & Ac, Gy, Li, Mg & 95.6-98.1\% \\

        \hline
        \multicolumn{4}{p{12cm}}{\footnotesize \textbf{Ac}: Accelerometer, \textbf{Ca}: Camera,  \textbf{Gr}: Gravity Sensor, \textbf{Gy}: Gyroscope, \textbf{Li}: Light Sensor, \textbf{Mg}: Magnetometer,  \textbf{Mi}: Microphone, \textbf{Nw}: Network, \textbf{Or}: Orientation, \textbf{To}: Touch, \textbf{Tr}: Tracking} \\
        \hline
        
        \multicolumn{4}{p{12cm}}{\footnotesize \textbf{Be}: Behaviour, \textbf{Ga}: Gait, \textbf{Ge}: Gesture, \textbf{IaH}: In-air Handwriting, \textbf{Mo}: Motion}\\
        \hline
    \end{tabular}
    }
    \caption{Existing continuous authentication methods.}
    \label{tab:ca_approaches_deceivable}
\end{table}

\subsection{Spoofing: Achilles Heel of Continuous User Authentication}
\label{sec:continuous_authentication_mechanisms}

The conventional knowledge-based authentication approaches require a user to provide information like passwords to access their device. While these methods are simple to implement, they suffer from drawbacks such as the need for frequent re-entering and the possibility of reuse by an attacker. To mitigate these limitations, continuous authentication mechanisms have been implemented that verify the user's identity based on their behavioral biometrics such as keystroke dynamics, touch gestures, motion, and voice. By leveraging such inherent biometric signatures of the user, this approach aims to prevent unauthorized access. 
% Table \ref{tab:ca_approaches_deceivable} offers an overview of different continuous authentication mechanisms across five widely used modalities. 

The user's biometric signatures are captured implicitly through various data streams, including interaction, environmental information, and sensory data. Hence, the underlying assumption to justify the security of these mechanisms is that these biometric signatures are unique to the user and difficult for attackers to replicate. Upon analyzing these approaches, we determined that the majority of them rely on user data (specifically, sensor data) as the primary data source for user authentication. Hence, these authentication mechanisms can be bypassed by manipulating the user data (especially sensor data) fed into them. The attacker can simply present false sensor data replicating the data recorded when the authentic user was utilizing the device. This can lead the authentication mechanism to erroneously conclude that the genuine user is using the device and allow the attacker to gain unauthorized access to sensitive data or perform actions on behalf of the legitimate user. 

However, we note that since there is no robust mechanism for spoofing the user data in the existing literature, the aforementioned attack on the continuous authentication mechanisms is difficult to realize in the real world. The absence of such a strong attack has resulted in the wide acceptance of weak continuous authentication methods~\cite{kolokas2019gait, sun2018artificial, thang2012gait, hoang2013adaptive, shih2015flick, nohara2016personal, lu2015safeguard, jain2015exploring, nixon2016slowmo, feng2014tips, abuhamad2020autosen, amini2018deepauth, li2018using, yan2018towards, song2016eyeveri, xia2018motionhacker, hong2016mgra, hong2015waving, miguel2016interaction, zhang2016voicelive, wang2019voicepop, johnson2013secure, khamis2016gazetouchpass, zhu2013sensec, sitova2015hmog, pang2019mineauth, acien2019multilock, zhu2019riskcog, lee2017implicit}.

To mitigate the insufficiency of the Android permission framework and to establish a practical benchmark for continuous authentication mechanisms, there is a need for a more robust and comprehensive approach to spoof user data in the Android ecosystem. This paper aims to address this critical gap by identifying and addressing the limitations of existing data spoofing mechanisms.