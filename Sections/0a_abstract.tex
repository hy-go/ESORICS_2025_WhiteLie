\begin{abstract}
    Android employs a permission framework that empowers users to either accept or deny sharing their private data (e.g., location) with an app. However, many apps tend to crash when they are denied permission, leaving users no choice but to allow access to their data in order to use the app. 
    % However, existing approaches either require rooting the device or require recompiling the app. 
    In this paper, we introduce a comprehensive and robust user data spoofing system, \framework{}, that can spoof a variety of user data and feed it to target apps. Unlike prior approaches, \framework{} requires neither device rooting nor altering the app's binary. Through experiments on more than 50 popular Android apps, we demonstrate that \framework{} is able to deceive apps into accepting spoofed data without getting detected. \framework{} does not significantly impact battery consumption, CPU utilization, or the app's execution time. We also show several case studies where \framework{} is useful in protecting user privacy. Our findings underscore the feasibility of implementing user-centric privacy-enhancing mechanisms within the existing Android ecosystem.
    
    \keywords{Android \and User Data Privacy \and Continuous Authentication \and Side-Channel Attacks.}
\end{abstract}