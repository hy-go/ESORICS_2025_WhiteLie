\section{Related Work}
\label{sec:related_work}

Several past works~\cite{bokhorst2017xprivacy,bokhorst2021xprivacylua,hornyack2011these,ratazzi2019pinpoint,raval2019permissions,shrestha2016slogger} block or filter the user data to protect user's privacy. However, by spoofing the data instead of outright blocking or filtering it, provides users with greater control over their data without crashing the app.

Many approaches~\cite{backes2015boxify,jeon2012dr,raval2016you,smalley2013security,wu2017context} modify the Android OS or third-party app source code to deceive user data fed into apps. However, source code modification is detectable in modern Android OS making the tactic ineffective. In contrast, \framework{} eliminates the need for any modifications to OS or app's source code.

Other approaches enhance the Android Permission Framework by providing intent transparency for third-party apps~\cite{chitkara2017does,tsai2017turtle,conti2011crepe,thanigaivelan2018codra}, temporarily revoking permissions based on user policies~\cite{bugiel2013flexible,liu2016follow,chakraborty2014ipshield}, app activities~\cite{zhang2015leave,chen2013contextual,chen2017sweetdroid,petracca2015audroid}, or machine learning~\cite{olejnik2017smarper,rashidi2016android,wijesekera2017feasibility,fu2017inspired}. Revoking permissions from apps may cause crashes or limit functionality. A key future work in \framework{} is integrating context-based policies and machine learning to analyze app behavior via logs while spoofing user data, ensuring both a seamless user experience and enhanced privacy.
