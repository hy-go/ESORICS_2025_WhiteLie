\section{Conclusion}
\label{sec:conclusion}

In conclusion, this paper introduces \framework{}, a robust user data spoofing system designed to enhance user privacy without compromising app functionality on non-rooted Android devices with minimal overhead. Through extensive evaluations on popular apps \framework{} demonstrated its ability to effectively spoof 78.32\% of requested permissions without detection or app crashes. Beyond its primary purpose, \framework{} revealed its potential to bypass continuous authentication mechanisms based on sensor data, highlighting vulnerabilities in existing security measures. The findings of this research emphasize the importance of improving ongoing authentication methods and enhancing privacy safeguards and security measures within the Android ecosystem. The system also proved effective in detecting unexpected permission accesses by benign apps and mitigating side-channel attacks launched by malicious apps. Real-world examples, such as Facebook's unauthorized microphone access and protection against recently banned apps like Private SMS, showcase the diverse applications of \framework{}. 