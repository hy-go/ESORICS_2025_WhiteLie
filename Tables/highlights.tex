\begin{table}
    {\centering
    \begin{tabular} {>{\arraybackslash\centering}m{2.5cm} >{\arraybackslash}m{9.5cm}}
        \hline
         \textbf{Use Case} & \textbf{Example(s)}\\
         \hline
         Bypassing Continuous Authentication & 
         \textbf{MGRA}~\cite{hong2016mgra} (Sec \ref{sec:continuous_authentication_mechanisms}) utilizes the accelerometer data and \textbf{VoiceLive}~\cite{zhang2016voicelive} (Sec \ref{sec:continuous_authentication_mechanisms}) utilizes the smartphone's two different microphones to authenticate users. But \framework{} proves to be capable of bypassing these authentication mechanisms by spoofing the input data. \\
         \hline
         Granting Selective User Data & 
         \textbf{Snapchat}(\hyperref[sec:sc_case_study]{Sec 5.3}) and \textbf{Truecaller}(\hyperref[sec:tc_case_study]{Sec 5.4}) requires users private data like Contacts and Messages to enable users to utilize functionalities provided by them, but this also compromises the private information. However, \framework{} enables users to grant partial data to apps that is essential.\\
         \hline
         Protection from Malicious Apps & 
         \textbf{All Good PDF Scanner} (Sec \ref{sec:malicious_apps}) was granted permissions like Storage and Camera but was detected to be uploading data on the internet in the background, hence its internet access was blocked by \framework{}. \textbf{Unique Keyboard} (Sec \ref{sec:malicious_apps}) was found to be accessing the sensor data while running in the background, therefore \framework{} fed deceived data when app is running in the background.\\
         \hline
         Side-Channel Attack Mitigation & 
         \textbf{Gyrosec}~\cite{lin2019motion} (Sec \ref{sec:side_channel_attack}) records the sensor data while running in the background, and uploads it over the internet. This was mitigated by deceiving the sensor data received by the apps.  \\
         \hline
         Unexpected Sensor Usage Detection & 
         \textbf{Facebook} (Sec \ref{sec:fb_case_study}) requested for Audio permission. But it was found (using \framework{}) to be recording the microphone data when no activity requiring microphone was being used by the user.\\
         \hline
         User Privacy Protection & 
         \textbf{Snapchat} (Sec \ref{sec:sc_case_study}) and \textbf{Truecaller} (Sec \ref{sec:tc_case_study}) provide great features to users but at the cost of sharing private information like Location, Contacts and Messages. \framework{} enables users to enjoy the provided features without compromising private information.\\
         \hline
    \end{tabular}
    }
    \caption{Various use cases of \framework{} and their respective real-world example(s)}
    \label{tab:highlights}
\end{table}